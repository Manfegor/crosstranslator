\Chapter{Szintaktikai elemzés}

% TODO: Formális nyelvekkel, fordítóprogramokkal kapcsolatos könyvek hivatkozásai.

% http://www.informatik.uni-bremen.de/agbkb/lehre/ccfl/Material/ALSUdragonbook.pdf

Ennek a célja az, hogy megállapítsa, hogy érvényes programról van-e szó, illetve hogy majd egyszerűen fa struktúrába rendezhető legyen a program. (AST és CST problémaköre)

Ide kerül a szintaktikai elemzés elemzés elterjedt módjainak a bemutatása. Elvi szinten

\subsection{Példa az összehasonlításhoz}

% A nyelvtan felírása EBNF-fel

% A nyelvtan felrajzolása szintaxis diagrammal

% A nyelv egy szava ~ kvázi egy forráskód részlet 8-10 sorral.

\section{Java parzer generátorok}

% TODO: Átnézni az elterjedt generátorokat: https://en.wikipedia.org/wiki/Comparison_of_parser_generators

\subsection{AnnoFlex}

% https://github.com/annoflex/annoflex

\subsection{JFlex}

% http://jflex.de/

\subsection{...}

% TODO: Összeszedni 4-5 használhatót, és csinálni hozzá példákat!
