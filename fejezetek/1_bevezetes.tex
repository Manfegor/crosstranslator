\Chapter{Bevezetés}

A dolgozat egy olyan magas szintű programozási nyelv definiálását tűzi ki célul, mely fordítás után más, magas szintű programozási nyelvre fordul le. Ennek segítségével a programkódot nagyon könnyen lehet egyidejűleg több platformra is akár elkészíteni, hiszen a program egy adott nyelven történő megírása után a fordítás során több nyelvre, több platfomra lehet fordítani.
A dolgozat elkészítése során fontos szempont volt megvizsgálni a célnyelvkent elérni kívánt programozási nyelveket, azok szintaxisát, azokat összehasonlítani, hogy a hasonlóságok és különbségek kiemelése révén megfogalmazható legyen, hogy milyen legyen a definiált nyelv.
Emellett mindenképpen vizsgálni kellett a programkód feldolgozásának lépéseit, a parser és lexer működését. A következő lépés pedig az eddig megszerzett ismereteit alapján az elkészíteni kívánt nyelv tulajdonságait, vezérlési szerkezeteit, definiálni az elemeit.

% Leírni, hogy milyen problémát és hogy old meg a dolgozat. Hasonló, mint a feladatkiírás, viszont itt a cél az, hogy aki ezt elolvassa, az kedvet kapjon a többi részhez is.

% Itt hangsúlyozni kell a feladat és a megoldás létjogosultságát, tehát hogy így nagyon egyszerűen lehet egy kódot egyidejűleg több platformra is elkészíteni.